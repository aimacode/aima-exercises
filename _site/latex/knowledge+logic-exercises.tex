\setlength{\medskipamount}{1.25\medskipamount}%%%%% Mona's suggestion

%%%% 7.2: The Wumpus World (1 exercises, 0 labelled)
%%%% ===============================================

\begin{exercise}
Suppose the agent has progressed to the point shown in
\figref{wumpus-seq35-figure}(a), \pgref{wumpus-seq35-figure}, having perceived nothing in [1,1], a
breeze in [2,1], and a stench in [1,2], and is now concerned with the
contents of [1,3], [2,2], and [3,1]. Each of these can contain a pit,
and at most one can contain a wumpus. Following the example of
\figref{wumpus-entailment-figure}, construct the set of possible
worlds. (You should find 32 of them.)  Mark the worlds in which the KB
is true and those in which each of the following sentences is true:
\begin{formula}\zt 
\alpha_2 = \mbox{``There is no pit in [2,2].''}\\\zt 
\alpha_3 = \mbox{``There is a wumpus in [1,3].''}
\end{formula}
Hence show that \(\J{KB} \entails \alpha_2\) and \(\J{KB} \entails \alpha_3\).
\end{exercise} 
% id=7.0 section=7.2


%%%% 7.3: Logic (1 exercises, 0 labelled)
%%%% ====================================

\begin{exercise}
(Adapted from \citeA{Barwise+Etchemendy:1993}.)  Given the following,
can you prove that the unicorn\index{unicorn} is mythical?  How about
magical?  Horned?
\begin{quote}
If the unicorn is mythical, then it is immortal, but if it is not mythical,
then it is a mortal mammal.  If the unicorn is either immortal or a mammal,
then it is horned.  The unicorn is magical if it is horned.
\end{quote}
\end{exercise} 
% id=7.10 section=7.3


%%%% 7.4: Propositional Logic: A Very Simple Logic (10 exercises, 5 labelled)
%%%% ========================================================================

\begin{exercise}[truth-value-exercise]
Consider the problem of deciding whether a propositional logic sentence is 
true in a given model.
\begin{enumerate}
 \item Write a recursive algorithm
 \prog{PL-True?}(\var{s}{\ac}\var{m}) that returns \var{true} if and
 only if the sentence \(s\) is true in the model \(m\) (where \(m\) assigns
 a truth value for every symbol in \(s\)). The algorithm should run in
 time linear in the size of the sentence. (Alternatively, use a version
 of this function from the online code repository.)
\item Give three examples of sentences that can be determined to be true or false in a {\em partial} model
that does not specify a truth value for some of the symbols.
\item Show that the truth value (if any) of a sentence in a partial model
cannot be determined efficiently in general.
\item Modify your \prog{PL-True?} algorithm so that it can sometimes judge truth
from partial models, while retaining its recursive structure and linear run time.
Give three examples of sentences whose truth in a partial model is {\em not} detected by your algorithm.
\item Investigate whether the modified algorithm makes \prog{TT-Entails?} more efficient.
\end{enumerate}
\end{exercise} 
% id=7.1 section=7.4.4

\begin{uexercise}%% Russell Fall 2002 final
Which of the following are correct?
\begin{enumerate}
\item \(\J{False} \models \J{True}\). 
\item \(\J{True} \models \J{False}\). 
\item \((A\land B)  \models (A\lequiv B)\). 
\item \(A\lequiv B \models A \lor B\).
\item \(A\lequiv B \models \lnot A \lor B\).
\item \((A\land B)\implies C \models (A\implies C)\lor(B\implies C)\).
\item \((C\lor (\lnot A \land \lnot B)) \equiv ((A\implies C) \land (B \implies C))\).
\item \((A\lor B) \land (\lnot C\lor\lnot D\lor E) \models (A\lor B)\).
\item \((A\lor B) \land (\lnot C\lor\lnot D\lor E) \models (A\lor B) \land (\lnot D\lor E)\).
\item \((A\lor B) \land \lnot(A \implies B)\) is satisfiable.
\item \((A\lequiv B) \land (\lnot A \lor B)\) is satisfiable.
\item \((A\lequiv B) \lequiv C\) has the same number of models as \((A\lequiv B)\) for any fixed
set of proposition symbols that includes \(A\), \(B\), \(C\).
\end{enumerate}
\end{uexercise} 
% id=7.2 section=7.4.2

\begin{iexercise}%% Russell Fall 2002 final
Which of the following are correct?
\begin{enumerate}
\item \(\J{False} \models \J{True}\). 
\item \(\J{True} \models \J{False}\). 
\item \((A\land B)  \models (A\lequiv B)\). 
\item \(A\lequiv B \models A \lor B\).
\item \(A\lequiv B \models \lnot A \lor B\).
\item \((A\lor B) \land (\lnot C\lor\lnot D\lor E) \models (A\lor B\lor C) \land (B\land C\land D\implies E)\).
\item \((A\lor B) \land (\lnot C\lor\lnot D\lor E) \models (A\lor B) \land (\lnot D\lor E)\).
\item \((A\lor B) \land \lnot(A \implies B)\) is satisfiable.
\item \((A\land B)\implies C \models (A\implies C)\lor(B\implies C)\).
\item \((C\lor (\lnot A \land \lnot B)) \equiv ((A\implies C) \land (B \implies C))\).
\item \((A\lequiv B) \land (\lnot A \lor B)\) is satisfiable.
\item \((A\lequiv B) \lequiv C\) has the same number of models as \((A\lequiv B)\) for any fixed
set of proposition symbols that includes \(A\), \(B\), \(C\).
\end{enumerate}
\end{iexercise} 
% id=7.2 section=7.4.2

\begin{exercise}[deduction-theorem-exercise]
Prove each of the following assertions:
\begin{enumerate}
\item \(\alpha\) is valid if and only if \(\J{True}\entails \alpha\).
\item For any \(\alpha\), \(\J{False}\entails\alpha\).
\item \(\alpha\entails \beta\) if and only if the sentence \((\alpha \implies \beta)\) is valid.
\item \(\alpha \equiv \beta\) if and only if the sentence \((\alpha\lequiv\beta)\) is valid.
\item \(\alpha\entails \beta\) if and only if the sentence \((\alpha \land \lnot \beta)\) is unsatisfiable.
\end{enumerate}
\end{exercise} 
% id=7.3 section=7.4.2

\begin{iexercise}%% Russell Fall 2005 final, Spring 2002 final, Spring 2004 final
Prove, or find a counterexample to, each of the following assertions:
\begin{enumerate}
\item  If \(\alpha\models\gamma\) or \(\beta\models\gamma\) (or both) then \((\alpha\land \beta)\models\gamma\)
\item  If \((\alpha\land \beta)\models\gamma\) then \(\alpha\models\gamma\) or \(\beta\models\gamma\) (or both).
\item  If \(\alpha\models (\beta \lor \gamma)\) then \(\alpha \models \beta\) or \(\alpha \models \gamma\) (or both).
\end{enumerate}
\end{iexercise} 
% id=7.4 section=7.4.2

\begin{uexercise}%% Russell Fall 2005 final, Spring 2002 final, Spring 2004 final
Prove, or find a counterexample to, each of the following assertions:
\begin{enumerate}
\item  If \(\alpha\models\gamma\) or \(\beta\models\gamma\) (or both) then \((\alpha\land \beta)\models\gamma\)
\item  If \(\alpha\models (\beta \land \gamma)\) then \(\alpha \models \beta\) and \(\alpha \models \gamma\).
\item  If \(\alpha\models (\beta \lor \gamma)\) then \(\alpha \models \beta\) or \(\alpha \models \gamma\) (or both).
\end{enumerate}
\end{uexercise} 
% id=7.4 section=7.4.2

\begin{exercise}
Consider a vocabulary with only four propositions, \(A\), \(B\), \(C\), and
\(D\).  How many models are there for the following sentences?
\begin{enumerate}
\item \(B\lor C\).
\item \(\lnot A\lor \lnot B \lor \lnot C \lor \lnot D\).
\item \((A\implies B) \land A \land \lnot B \land C \land D\).
\end{enumerate}
\end{exercise} 
% id=7.6 section=7.4.2

\begin{exercise}
We have defined four  binary logical connectives.  
\begin{enumerate}
\item Are there any others that might be useful?  
\item How many binary connectives can there be?  
\item Why are some of them not very useful?
\end{enumerate}
\end{exercise} 
% id=7.7 section=7.4

\begin{exercise}[logical-equivalence-exercise]%
Using a method of your choice, verify each of the equivalences
in \tabref{logical-equivalence-table} (\pgref{logical-equivalence-table}).
\end{exercise} 
% id=7.8 section=7.4

\begin{uexercise}[propositional-validity-exercise]%
Decide whether each of the following sentences is valid,
unsatisfiable, or neither.  Verify your decisions using truth tables
or the equivalence rules of \tabref{logical-equivalence-table} (\pgref{logical-equivalence-table}).
%%<<consider some new ones]]
\begin{enumerate}
\item \(\J{Smoke} \implies \J{Smoke}\)
\item \(\J{Smoke} \implies \J{Fire}\)
\item \((\J{Smoke} \implies \J{Fire}) \implies (\lnot \J{Smoke} \implies \lnot \J{Fire})\)
\item \(\J{Smoke} \lor \J{Fire} \lor \lnot \J{Fire}\)
\item \(((\J{Smoke} \land \J{Heat}) \implies \J{Fire}) 
        \lequiv ((\J{Smoke} \implies \J{Fire}) \lor (\J{Heat} \implies \J{Fire}))\)
\item \((\J{Smoke} \implies \J{Fire}) \implies 
        ((\J{Smoke} \land \J{Heat}) \implies \J{Fire}) \)
\item \(\J{Big} \lor \J{Dumb} \lor (\J{Big} \implies \J{Dumb})\)
\end{enumerate}
\end{uexercise} 
% id=7.9 section=7.4

\begin{iexercise}[propositional-validity-exercise]%
Decide whether each of the following sentences is valid,
unsatisfiable, or neither.  Verify your decisions using truth tables
or the equivalence rules of \tabref{logical-equivalence-table} (\pgref{logical-equivalence-table}).
%%<<consider some new ones]]
\begin{enumerate}
\item \(\J{Smoke} \implies \J{Smoke}\)
\item \(\J{Smoke} \implies \J{Fire}\)
\item \((\J{Smoke} \implies \J{Fire}) \implies (\lnot \J{Smoke} \implies \lnot \J{Fire})\)
\item \(\J{Smoke} \lor \J{Fire} \lor \lnot \J{Fire}\)
\item \(((\J{Smoke} \land \J{Heat}) \implies \J{Fire}) 
        \lequiv ((\J{Smoke} \implies \J{Fire}) \lor (\J{Heat} \implies \J{Fire}))\)
\item \(\J{Big} \lor \J{Dumb} \lor (\J{Big} \implies \J{Dumb})\)
\item \((\J{Big} \land \J{Dumb}) \lor \lnot \J{Dumb}\)
\end{enumerate}
\end{iexercise} 
% id=7.9 section=7.4

%% \begin{iexercise}
%% Describe how the map-coloring problem, described in \secref{map-coloring-section}, can be expressed in
%% propositional logic. ({\em Hint}: As your proposition symbols, use assignments
%% of one color to one region;for example, \(Q\_R\) means that Queensland
%% is colored red.)
%% \end{iexercise} 
% id=7.18 section=7.4

\begin{exercise}[cnf-proof-exercise]
Any propositional logic sentence is logically equivalent to the 
assertion that each possible world in which it would be false 
is not the case. From this observation, prove that any sentence can be written in CNF.
%% The negation of each row is the negation of a conjunction of literals,
%% which (by de Morgan's law) is equivalent to a disjunction
%% of the negations of literals, which is equivalent to a disjunction of
%% literals.
\end{exercise} 
% id=7.23 section=7.4


%%%% 7.5: Propositional Theorem Proving (9 exercises, 1 labelled)
%%%% ============================================================

\begin{exercise}
Use resolution to prove the sentence \(\lnot A \land \lnot B\) from the clauses in
\exref{convert-clausal-exercise}.
\end{exercise} 
% id=7.12 section=7.5.2

\begin{exercise}[inf-exercise]
This exercise looks into the relationship between clauses and implication sentences.
\begin{enumerate}
\item Show that the clause \((\lnot P_1 \lor \cdots \lor \lnot P_m \lor Q)\)
is logically equivalent to the implication sentence \((P_1 \land \cdots \land P_m) \textimplies Q\).
\item Show that every clause (regardless of the number of positive literals) can be written
in the form \((P_1 \land \cdots \land P_m) \textimplies (Q_1 \lor \cdots \lor Q_n)\),
where the \(P\)s and \(Q\)s are proposition symbols.
A knowledge base consisting of such sentences is 
in \newterm{implicative normal form}\ntindex{implicative normal form}
or \termi{Kowalski form}  \cite{Kowalski:1979}.
\item Write down the full resolution rule for sentences in implicative normal form.
\end{enumerate}
\end{exercise} 
% id=7.15 section=7.5.3

\begin{exercise}%% Russell Spring 2004 midterm
According to some political pundits, a person who is radical (\(R\)) is electable (\(E\))
if he/she is conservative (\(C\)), but otherwise is not electable.
\begin{enumerate}
\item  Which of the following are correct representations of this assertion?
\begin{enumerate}
\item \((R\land E)\iff C\) 
\item \(R\implies (E\iff C)\) 
\item \(R\implies ((C\implies E) \lor \lnot E)\) 
\end{enumerate}
\item  Which of the sentences in (a) can be expressed in Horn form?
\end{enumerate}
\end{exercise} 
% id=7.16 section=7.5.3

\begin{uexercise}%% Russell Spring 2005 final
This question considers representing satisfiability
(SAT) problems as CSPs.
\begin{enumerate}
\item  Draw the constraint graph corresponding to the SAT problem
\[
  (\lnot X_1 \lor X_2) \land (\lnot X_2 \lor X_3) \land \ldots \land (\lnot X_{n-1} \lor X_n)
\]
for the particular case \(n\eq 5\).
\item  How many solutions are there for this general SAT problem as a function of \(n\)?
\item  Suppose we apply \prog{Backtracking-Search} (\pgref{backtracking-search-algorithm}) to find {\em all} solutions
to a SAT CSP of the type given in (a). (To find {\em all} solutions to a CSP, we simply modify
the basic algorithm so it continues searching after each solution is found.)
Assume that variables are ordered \(X_1,\ldots,X_n\) and \(\J{false}\) is ordered before \(\J{true}\).
How much time will the algorithm take to terminate? (Write an \(O(\cdot)\) expression as a function of \(n\).)
\item  We know that SAT problems in Horn form can be solved in linear time by forward chaining (unit propagation).
We also know that every tree-structured binary CSP with discrete, finite domains can be solved in time linear in the number of variables (\secref{csp-structure-section}).
Are these two facts connected? Discuss.
\end{enumerate}
\end{uexercise} 
% id=7.17 section=7.5

\begin{iexercise}%% Russell Spring 2005 final
This question considers representing satisfiability
(SAT) problems as CSPs.
\begin{enumerate}
\item  Draw the constraint graph corresponding to the SAT problem
\[
  (\lnot X_1 \lor X_2) \land (\lnot X_2 \lor X_3) \land \ldots \land (\lnot X_{n-1} \lor X_n)
\]
for the particular case \(n\eq 4\).
\item  How many solutions are there for this general SAT problem as a function of \(n\)?
\item  Suppose we apply \prog{Backtracking-Search} (\pgref{backtracking-search-algorithm}) to find {\em all} solutions
to a SAT CSP of the type given in (a). (To find {\em all} solutions to a CSP, we simply modify
the basic algorithm so it continues searching after each solution is found.)
Assume that variables are ordered \(X_1,\ldots,X_n\) and \(\J{false}\) is ordered before \(\J{true}\).
How much time will the algorithm take to terminate? (Write an \(O(\cdot)\) expression as a function of \(n\).)
\item  We know that SAT problems in Horn form can be solved in linear time by forward chaining (unit propagation).
We also know that every tree-structured binary CSP with discrete, finite domains can be solved in time linear in the number of variables (\secref{csp-structure-section}).
Are these two facts connected? Discuss.
\end{enumerate}
\end{iexercise} 
% id=7.17 section=7.5

\begin{uexercise}%% Russell Spring 2005 midterm
Explain why every nonempty propositional clause, by itself, is
satisfiable.  Prove rigorously that every set of five 3-SAT clauses is
satisfiable, provided that each clause mentions exactly three distinct
variables. What is the smallest set of such clauses that is unsatisfiable?
Construct such a set.
\end{uexercise} 
% id=7.19 section=7.5

\begin{uexercise}%% Russell Fall 2002 midterm
A propositional {\em 2-CNF} expression is a conjunction of clauses,
each containing {\em exactly 2} literals, e.g., 
\[
  (A\lor B) \land (\lnot A \lor C) \land (\lnot B \lor D) \land (\lnot
  C \lor G) \land (\lnot D \lor G)\ .
\]
\begin{enumerate}
\item  Prove using resolution that the above sentence entails \(G\).
\item  Two clauses are {\em semantically distinct} if they are not logically equivalent.
How many semantically distinct 2-CNF clauses can be constructed from \(n\) proposition symbols?
\item  Using your answer to (b), prove that propositional resolution always
terminates in time polynomial in \(n\) given a 2-CNF sentence containing no more than \(n\) distinct symbols.
\item  Explain why your argument in (c) does not apply to 3-CNF.
\end{enumerate}
\end{uexercise} 
% id=7.20 section=7.5.2

\begin{iexercise}%% Russell Fall 2005 midterm
Prove each of the following assertions:
\begin{enumerate}
\item Every pair of propositional clauses either has no resolvents,
or all their resolvents are logically equivalent.
\item  There is no clause that, when resolved with itself, yields
(after factoring) the clause \((\lnot P \lor \lnot Q)\).
\item  If a propositional clause \(C\) can be resolved with a copy
of itself, it must be logically equivalent to \(\J{True}\).
\end{enumerate}
\end{iexercise} 
% id=7.21 section=7.5.2

\begin{exercise}%% Russell Spring 2002
Consider the following sentence:
\[
  [ (\J{Food} \implies \J{Party}) \lor (\J{Drinks} \implies \J{Party}) ] \implies [ ( \J{Food} \land \J{Drinks} )  \implies \J{Party}]\ .
\]
\begin{enumerate}
\item  Determine, using enumeration,  whether this sentence is valid, satisfiable (but not valid), or unsatisfiable.

\item  Convert the left-hand and right-hand sides of the main implication into CNF, showing each step,
and explain how the results confirm your answer to (a).
\item  Prove your answer to (a) using resolution.
\end{enumerate}
\end{exercise} 
% id=7.22 section=7.5.2

\begin{exercise}\label{dnf-exercise}
A sentence is in \newtermi{disjunctive normal form} (DNF)\index{DNF (disjunctive normal form)} if it is the disjunction
of conjunctions of literals.  For example, the sentence \((A \land B \land \lnot C) \lor (\lnot A \land C) \lor (B \land \lnot C)\) is in DNF.
\begin{enumerate}
\item Any propositional logic sentence is logically equivalent to the 
assertion that some possible world in which it would be true is
in fact the case. From this observation, prove that any sentence can be written in DNF.
\item Construct an algorithm that converts any sentence in 
propositional logic into DNF. ({\em Hint}: The algorithm is 
similar to the 
algorithm for conversion to CNF given in \secref{pl-resolution-section}.)
\item
Construct a simple algorithm that takes as input a sentence in DNF and returns
a satisfying assignment if one exists, or reports that no satisfying assignment
exists.
\item
Apply the algorithms in (b) and (c) to the following set of sentences:
\begin{formula}
A \implies B \\
B \implies C \\
C \implies \lnot A\ .
\end{formula}
\item Since the algorithm in (b) is very similar to the 
algorithm for conversion to CNF, and since the algorithm in (c) is much simpler 
than any algorithm for solving a set of sentences in CNF, why is this technique
not used in automated reasoning?
\end{enumerate}
\end{exercise} 
% id=7.24 section=7.5


%%%% 7.6: Effective Propositional Model Checking (6 exercises, 4 labelled)
%%%% =====================================================================

%% \begin{iexercise}%% Russell Spring 2004 final
%% Two sentences \(\alpha\) and \(\beta\) are said to be logically \newterm[consistency]{consistent}\ntindex{consistency!logical} if neither entails the falsehood of the other.
%% \begin{enumerate}
%% \item  Explain how to determine the consistency of any two propositional sentences by using a SAT solver.
%% \item  Give examples of propositional CNF sentences \(\alpha\) and \(\beta\), each containing the same \(n\)
%% proposition symbols, such that neither \(\alpha \models \beta\) nor \(\alpha\models\lnot \beta\).
%% Prove that this cannot be done if \(\alpha\) consists only of unit clauses. 
%% \item  Is every sentence consistent with itself?
%% \end{enumerate}
%% \end{iexercise} 
%% % id=7.5 section=7.6

\begin{uexercise}[convert-clausal-exercise]
Convert the following set of sentences to clausal form.
\begin{quote}
S1: \(A \lequiv (B \lor E)\). \\
S2: \(E \implies D\). \\
S3: \(C \land F \implies \lnot B\). \\
S4: \(E \implies B\). \\
S5: \(B \implies F\). \\
S6: \(B \implies C\)
\end{quote}
Give a trace of the execution of DPLL on the conjunction of
these clauses.  
\end{uexercise} 
% id=7.11 section=7.6.1

\begin{iexercise}[convert-clausal-exercise]
Convert the following set of sentences to clausal form.
\begin{quote}
S1: \(A \lequiv (C \lor E)\). \\
S2: \(E \implies D\). \\
S3: \(B \land F \implies \lnot C\). \\
S4: \(E \implies C\). \\
S5: \(C \implies F\). \\
S6: \(C \implies B\)
\end{quote}
Give a trace of the execution of DPLL on the conjunction of
these clauses.  
\end{iexercise} 
% id=7.11 section=7.6.1

\begin{exercise}
Is a randomly generated 4-CNF sentence with \(n\) symbols and \(m\) clauses more or
less likely to be solvable than 
a randomly generated 3-CNF sentence with \(n\) symbols and \(m\) clauses? Explain.
\end{exercise} 
% id=7.13 section=7.6.3

\begin{exercise}[minesweeper-exercise]
Minesweeper, the well-known computer game, is closely related to the
wumpus world. A \indextext{minesweeper} world is a rectangular grid
of \(N\) squares with \(M\) invisible mines scattered among them.
Any square may be probed by the agent; instant death follows if a mine
is probed. Minesweeper indicates the
presence of mines by revealing, in each probed square, the {\em
number} of mines that are directly or diagonally adjacent.
The goal is to  probe every unmined square.
\begin{enumerate}
\item Let \(X_{i,j}\) be true iff square \([i,j]\) contains a mine.
Write down the assertion that exactly two mines are adjacent to
[1,1] as a sentence involving some logical combination of \(X_{i,j}\) propositions.
\item Generalize your assertion from (a)
 by explaining how to construct a CNF sentence
asserting that \(k\) of \(n\) neighbors contain mines.
\item Explain precisely how an agent can use \prog{DPLL} to prove
that a given square does (or does not) contain a mine, ignoring the
global constraint that there are exactly \(M\) mines in all.
\item Suppose that the global constraint is constructed
from your method from part (b). How does the number of clauses
depend on \(M\) and \(N\)? Suggest a way to modify \prog{DPLL} so that
the global constraint does not need to be represented explicitly.
\item Are any conclusions derived by the method in part (c)
invalidated when the global constraint is taken into account?
\item Give examples of configurations of probe values that induce {\em
long-range dependencies} such that the contents of a
given unprobed square would give information about the contents of a
far-distant square. ({\em Hint}: consider an \(N\stimes 1\) board.)
\end{enumerate}
\end{exercise} 
% id=7.14 section=7.6

\begin{exercise}[known-literal-exercise]
How long does it take to prove \(\J{KB}\entails\alpha\) using \prog{DPLL} when \(\alpha\) is a literal
{\em already contained in} \(\J{KB}\)? Explain.
\end{exercise} 
% id=7.25 section=7.6.1

\begin{exercise}[dpll-fc-exercise]
Trace the behavior of \prog{DPLL} on the knowledge base
in \figref{pl-horn-example-figure} when trying to prove \(Q\), and
compare this behavior with that of the forward-chaining algorithm.
\end{exercise} 
% id=7.26 section=7.6.1


%%%% 7.7: Agents Based on Propositional Logic (5 exercises, 2 labelled)
%%%% ==================================================================

\begin{uexercise} % fe-s04
Write a successor-state axiom for the \(\J{Locked}\) predicate, 
which applies to doors, assuming the only actions available are \(\J{Lock}\)
and \(\J{Unlock}\).
\end{uexercise} 
% id=extras-28-oct.8 section=7.7.1

\begin{iexercise}
Discuss what is meant by {\em optimal} behavior in the wumpus world.
Show that the \prog{Hybrid-Wumpus-Agent} is not optimal,
and suggest ways to improve it.
\end{iexercise} 
% id=7.27 section=7.7.2

\begin{iexercise}%% Russell Fall 2005 final
Suppose an agent inhabits a world with two states, \(S\) and \(\lnot S\), and
can do exactly one of two actions, \(a\) and \(b\). Action \(a\) does nothing and action \(b\) flips from
one state to the other. Let \(S^t\) be the proposition that the agent is in state \(S\) at time \(t\),
and let \(a^t\) be the proposition that the agent does action \(a\) at time \(t\) (similarly for \(b^t\)).
\begin{enumerate}
\item  Write a successor-state axiom for \(S^{t+1}\).
\item  Convert the sentence in (a) into CNF.
\item  Show a resolution refutation proof that if the agent is in \(\lnot S\) at time \(t\) and does \(a\), 
it will still be in \(\lnot S\) at time \(t+1\).
\end{enumerate}
\end{iexercise} 
% id=7.28 section=7.7

\begin{exercise}[ss-axiom-exercise]
\secref{successor-state-section} provides some of the successor-state
axioms required for the wumpus world. Write down axioms for all
remaining fluent symbols.
\end{exercise} 
% id=7.29 section=7.7

\begin{exercise}[hybrid-wumpus-exercise]\prgex%
Modify the \prog{Hybrid-Wumpus-Agent} to use the 1-CNF logical state estimation method described on 
\pgref{1cnf-belief-state-page}. We noted on that page that such an
agent will not be able to acquire, maintain, and use more complex
beliefs such as the disjunction \(P_{3,1}\lor P_{2,2}\). Suggest a
method for overcoming this problem by defining additional proposition
symbols, and try it out in the wumpus world. Does it improve the
performance of the agent?
\end{exercise} 
% id=7.30 section=7.7



%% \begin{exercise}\label{wumpus-dimensions-exercise}
%% Describe the wumpus world according to the properties of task environments
%% listed in \chapref{agents-chapter}. 
%% \end{exercise}

%% \begin{exercise}\label{wumpus-circuit-exercise}
%% In this exercise, you will design more of the circuit-based wumpus agent.
%% \begin{enumerate}
%% \item Write an equation, similar to \eqref{alive-equation}, for the
%% \(\J{Arrow}\) proposition, which should be true when the agent still has an arrow.
%% Draw the corresponding circuit.
%% \item Repeat part (a) for \(\J{FacingRight}\), using
%% \eqref{location-equation} as a model.
%% \item Create versions of Equations~\ref{knotpit-equation} and~\ref{kpit-equation} for finding the wumpus,
%% and draw the circuit.
%% \end{enumerate}
%% \end{exercise}




\resetmedskipamount
