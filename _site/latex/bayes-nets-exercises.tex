%%%% 14.2: The Semantics of Bayesian Networks (7 exercises, 3 labelled)
%%%% ==================================================================

\begin{uexercise}
We have a bag of three biased coins \(a\), \(b\), and \(c\) with probabilities of coming 
up heads of 20\%, 60\%, and 80\%, respectively.  One coin is drawn randomly from the bag (with equal 
likelihood of drawing each of the three coins), and then the coin is flipped three times to 
generate the outcomes \(X_1\), \(X_2\), and \(X_3\). 
\begin{enumerate}
\item Draw the Bayesian network corresponding to this setup and define the necessary 
CPTs.
\item Calculate which coin was most likely to have been drawn from the bag if the observed flips
come out heads twice and tails once.
\end{enumerate}
\end{uexercise} 
% id=14.0 section=14.2

\begin{iexercise}
We have a bag of three biased coins \(a\), \(b\), and \(c\) with probabilities of coming 
up heads of 30\%, 60\%, and 75\%, respectively.  One coin is drawn randomly from the bag (with equal 
likelihood of drawing each of the three coins), and then the coin is flipped three times to 
generate the outcomes \(X_1\), \(X_2\), and \(X_3\). 
\begin{enumerate}
\item Draw the Bayesian network corresponding to this setup and define the necessary 
CPTs.
\item Calculate which coin was most likely to have been drawn from the bag if the observed flips
come out heads twice and tails once.
\end{enumerate}
\end{iexercise} 
% id=14.0 section=14.2

\begin{exercise}[cpt-equivalence-exercise]
\eqref{parameter-joint-repn-equation} on \pgref{parameter-joint-repn-equation} defines the joint
distribution represented by a Bayesian network in terms of
the parameters \(\theta(X_i\given \Parents(X_i))\). This exercise asks you to derive the equivalence
between the parameters and the conditional probabilities \(\pv(X_i\given \Parents(X_i))\) from this definition.
\begin{enumerate}
\item Consider a simple network \(X\rightarrow Y\rightarrow Z\) with three Boolean variables. 
Use \eqrefs{conditional-probability-equation}{marginalization-equation} (\pgrefs{conditional-probability-equation}{marginalization-equation}) to express the conditional probability \(P(z\given y)\) as the ratio of two sums, each over entries in the joint distribution \(\pv(X,Y,Z)\).
\item Now use \eqref{parameter-joint-repn-equation} to write this expression in terms of the network parameters \(\theta(X)\), \(\theta(Y\given X)\), and \(\theta(Z\given Y)\).
\item Next, expand out the summations in your expression from part (b), writing out explicitly the terms for the true and false values of each summed variable.
Assuming that all network parameters satisfy the constraint \(\sum_{x_i} \theta(x_i\given \parents(X_i))\eq 1\), show that the resulting expression reduces to \(\theta(z\given y)\).
\item Generalize this derivation to show that \(\theta(X_i\given \Parents(X_i)) = \pv(X_i\given \Parents(X_i))\) for any Bayesian network.
\end{enumerate}
\end{exercise} 
% id=14.1 section=14.2

\begin{exercise}
The operation of \newtermi{arc reversal} in a Bayesian network allows us to change the direction of an 
arc \(X\rightarrow Y\) while preserving the joint probability distribution that the network represents~\cite{Shachter:1986}.  Arc reversal
may require introducing new arcs:  all the parents of \(X\) also become 
parents of \(Y\), and all parents of \(Y\) also become parents of \(X\).
\begin{enumerate}
\item Assume that \(X\) and \(Y\) start with \(m\) and \(n\) parents, respectively,
and that all variables have \(k\) values. By calculating the change in size for the CPTs of \(X\) and \(Y\),
show that the total number of parameters in the network cannot decrease during arc reversal. ({\em Hint}: the parents of
\(X\) and \(Y\) need not be disjoint.)
\item Under what circumstances can the total number remain constant?
\item Let the parents of \(X\) be \(\U \cup \V\) and the parents of \(Y\) be \(\V \cup \W\), where \(\U\) and \(\W\) are disjoint.
The formulas for the new CPTs after arc reversal are as follows:
\begin{eqnarray*}
\pv(Y\given \U,\V,\W) &=& \sum_x \pv(Y\given \V,\W, x) \pv(x\given \U, \V) \\
\pv(X\given \U,\V,\W, Y) &=& \pv(Y\given X, \V, \W) \pv(X\given \U, \V) / \pv(Y\given \U,\V,\W)\ .
\end{eqnarray*}
Prove that the new network expresses the same joint distribution over all variables as the original network.
\end{enumerate}
\end{exercise} 
% id=14.8 section=14.2

\begin{exercise}
Consider the Bayesian network in \figref{burglary-figure}.
\begin{enumerate}
\item If no evidence is observed, are \(\J{Burglary}\) and \(\J{Earthquake}\) independent?
Prove this from the numerical semantics and from the topological semantics.
\item If we observe \(\J{Alarm}\eq \J{true}\), are \(\J{Burglary}\) and \(\J{Earthquake}\) independent?
Justify your answer by calculating whether the probabilities involved satisfy the definition of conditional independence.
\end{enumerate}
\end{exercise} 
% id=14.9 section=14.2

\begin{uexercise}
Suppose that in a Bayesian network containing an unobserved variable \(Y\), all the 
variables in the Markov blanket \(\J{MB}(Y)\) have been observed.
\begin{enumerate}
\item Prove that removing the node \(Y\) from the network will not affect the 
posterior distribution for any other unobserved variable in the network.
\item Discuss whether we can remove \(Y\) if we are planning to use (i) rejection sampling and (ii) likelihood weighting.
\end{enumerate}
\end{uexercise} 
% id=14.10 section=14.2

\begin{figure}[h]
\threefigboxnew{figures/handedness1.eps}{figures/handedness2.eps}{figures/handedness3.eps}%%<<<upgrade figure]]
\caption{Three possible structures for a Bayesian network describing genetic inheritance of handedness.}
\label{handedness-figure}
\end{figure} 
% id=14.13 section=14.2

\begin{exercise}[handedness-exercise]
Let \(H_x\) be a random variable denoting the handedness of an individual \(x\), with possible values \(l\) or \(r\).
A common hypothesis is that left- or right-handedness is inherited by a simple mechanism; that is, perhaps there is
a gene \(G_x\), also with values \(l\) or \(r\), and perhaps actual handedness
turns out mostly the same (with some probability \(s\)) as the gene an individual possesses.
Furthermore, perhaps the gene itself is equally likely
to be inherited from either of an individual's parents, with a small nonzero probability \(m\)
of a random mutation flipping the handedness.  
\begin{enumerate}
\item Which of the three networks in \figref{handedness-figure} claim that 
\( \pv(G_{\J{father}},G_{\J{mother}},G_{\J{child}}) = \pv(G_{\J{father}})\pv(G_{\J{mother}})\pv(G_{\J{child}})\)?
\item Which of the three networks make independence claims that are consistent with the hypothesis about the inheritance of handedness?
\item Which of the three networks is the best description of the hypothesis?
\item Write down the CPT for the \(G_{\J{child}}\) node in network (a), in terms of \(s\) and \(m\).
\item Suppose that \(P(G_{\J{father}}\eq l)=P(G_{\J{mother}}\eq l)=q\). In network (a), derive an expression
for \(P(G_{\J{child}}\eq l)\) in terms of \(m\) and \(q\) only, by conditioning on its parent nodes.
\item Under conditions of genetic equilibrium, we expect the
distribution of genes to be the same across generations.
Use this to calculate the value of \(q\), and, given what you
know about handedness in humans, explain why the hypothesis described
at the beginning of this question must be wrong.
\end{enumerate}
\end{exercise} 
% id=14.14 section=14.2

\begin{exercise}[markov-blanket-exercise]
The \term{Markov blanket}\index{Markov!blanket} of a variable is defined on
\pgref{markov-blanket-page}. Prove that a variable is independent of all other variables in
the network, given its Markov blanket and derive \eqref{markov-blanket-equation} (\pgref{markov-blanket-equation}).
\end{exercise} 
% id=14.21 section=14.2


%%%% 14.3: Efficient Representation of Conditional Distributions (4 exercises, 2 labelled)
%%%% =====================================================================================

\begin{figure}[htp]
%%\epsfxsize=3.85in
\figboxnew{figures/car-starts.eps}
\caption{A Bayesian network describing some features of a car's electrical
system and engine. Each variable is Boolean, and the \(\J{true}\) value indicates
that the corresponding aspect of the vehicle is in working order.}
\label{car-starts-figure}
\end{figure} 
% id=14.2 section=14.3

\begin{exercise}
Consider the network for car diagnosis shown in \figref{car-starts-figure}. 
\begin{enumerate}
\item Extend the network with the Boolean variables \(\J{IcyWeather}\) and
\(\J{StarterMotor}\).
\item Give reasonable conditional probability tables for all the nodes.
\item How many independent values are contained in the joint
probability distribution for eight Boolean nodes, assuming that  no conditional
independence relations are known to hold among them?
\item How many independent probability values do your network tables contain?
\item The conditional distribution for \(\J{Starts}\) could be described
as a \termi{noisy-AND} distribution. Define this family in general and relate it to 
the noisy-OR distribution.
\end{enumerate}
\end{exercise} 
% id=14.3 section=14.3

\begin{iexercise}
Consider a simple Bayesian network with root variables \(\J{Cold}\),
\(\J{Flu}\), and \(\J{Malaria}\) and child variable \(\J{Fever}\),
with a noisy-OR conditional distribution for \(\J{Fever}\) as
described in \secref{canonical-distribution-section}. By adding
appropriate auxiliary variables for inhibition events and
fever-inducing events, construct an equivalent Bayesian network whose
CPTs (except for root variables) are deterministic. Define the CPTs
and prove equivalence.
\end{iexercise} 
% id=14.15 section=14.3

\begin{exercise}[LG-exercise]
Consider the family of linear Gaussian networks, as defined on
\pgref{LG-network-page}. 
\begin{enumerate}
\item In a two-variable network, let \(X_1\) be the parent of \(X_2\), let
\(X_1\) have a Gaussian prior, and let \(\pv(X_2\given X_1)\) be a linear Gaussian
distribution. Show that the joint distribution \(P(X_1,X_2)\) is a
multivariate Gaussian, and calculate its covariance matrix.
\item Prove by induction that the joint distribution for a general
linear Gaussian network on \(X_1,\ldots,X_n\) is also a multivariate Gaussian.
\end{enumerate}
\end{exercise} 
% id=14.16 section=14.3

\begin{exercise}[multivalued-probit-exercise]
The probit distribution defined on \pgref{probit-page} describes the
probability distribution for a Boolean child, given a single continuous
parent.
\begin{enumerate}
\item How might the definition be extended to cover multiple continuous parents?
\item How might it be extended to handle a {\em multivalued} child
variable? Consider both cases where the child's values are ordered (as
in selecting a gear while driving, depending on speed, slope, desired
acceleration, etc.) and cases where they are unordered (as in
selecting bus, train, or car to get to work). ({\em Hint}: Consider
ways to divide the possible values into two sets, to mimic a Boolean
variable.)
\end{enumerate}
\end{exercise} 
% id=14.17 section=14.3


%%%% 14.4: Exact Inference in Bayesian Networks (6 exercises, 3 labelled)
%%%% ====================================================================

\begin{exercise}
In your local nuclear\index{nuclear power} power station, there is an
alarm that senses when a temperature gauge exceeds a given
threshold. The gauge measures the temperature of the core.  Consider the
Boolean variables \(A\) (alarm sounds), \(F_A\) (alarm is faulty), and
\(F_G\) (gauge is faulty) and the multivalued nodes \(G\) (gauge reading)
and \(T\) (actual core temperature).
\begin{enumerate}
\item Draw a Bayesian network for this domain, given that the gauge is more likely to
fail when the core temperature gets too high.

\item Is your network a polytree? Why or why not?

\item Suppose there are just two possible actual and measured
temperatures, normal and high; the probability that the gauge gives
the correct temperature is \(x\) when it is working, but \(y\) when it is
faulty. Give the conditional probability table associated with \(G\).

\item  Suppose the alarm works correctly unless it is faulty, in which case it
never sounds. Give the conditional probability table associated with \(A\).

\item  Suppose the alarm and gauge are working and the alarm
sounds. Calculate an expression for
the probability that the temperature of the core is too high, in terms of
the various conditional probabilities in the network.

%% \item  In a given time period, the probability that the temperature
%% exceeds threshold is {\mathdelim}p{\mathdelim}. The cost of shutting down the reactor is {\mathdelim}c_s{\mathdelim}; the
%% cost of not shutting it down when the temperature is in fact too high is {\mathdelim}c_m{\mathdelim}
%% ({\mathdelim}m{\mathdelim} is for meltdown). Assuming the gauge and alarm to be working normally,
%% calculate the maximum value for {\mathdelim}x{\mathdelim} for which the gauge is of any use (i.e.,
%% if {\mathdelim}x{\mathdelim} is any higher than this, we have to shut down the reactor all the time).
%% 
%% \item  Suppose we add a second temperature gauge {\mathdelim}H{\mathdelim}, connected so
%% that the alarm goes off when either gauge reads High. Add {\mathdelim}H{\mathdelim} and {\mathdelim}F_H{\mathdelim}
%% (the event of {\mathdelim}H{\mathdelim} failing) to the network.
%%
%% \item  Are there circumstances under which adding a second gauge
%% would mean that we would need {\it more} accurate (i.e., more likely
%% to give the correct temperature) gauges? Why (not)?

\end{enumerate}
\end{exercise} 
% id=14.4 section=14.4

\begin{exercise}[telescope-exercise]
Two astronomers\index{astronomer} in different parts of the world make measurements
\(M_1\) and \(M_2\) of the number of stars \(N\) in some small region of the
sky, using their telescopes\index{telescope}. Normally, there is a small possibility \(e\)
of error by up to one star in each direction. Each telescope can also (with a much
smaller probability \(f\)) be badly out of focus (events \(F_1\) and \(F_2\)), in
which case the scientist will undercount by three or more stars (or
if \(N\) is less than 3, fail to detect any stars at all). Consider
the three networks shown in \figref{telescope-nets-figure}.
\begin{enumerate}
\item Which of these Bayesian networks  are correct (but not
necessarily efficient) representations of the preceding
information? 

\item Which is the best network? Explain.

\item Write out a conditional distribution for \(\pv(M_1\given N)\),
for the case where \(N\elt\{1,2,3\}\) and \(M_1\elt\{0,1,2,3,4\}\).
Each entry in the conditional distribution should be expressed
as a function of the parameters \(e\) and/or \(f\).

\item Suppose \(M_1\eq 1\) and \(M_2\eq 3\). What are the {\em possible} numbers
of stars if you assume no prior constraint on the values of \(N\)?

\item What is the {\em most likely} number of stars, given these observations?  
Explain how to compute this, or if it is not possible to compute, explain what additional
information is needed and how it would affect the result.
\end{enumerate}
\end{exercise} 
% id=14.5 section=14.4

\begin{uexercise}
Consider the network shown in \figref{telescope-nets-figure}(ii),
and assume that the two telescopes work identically.
\(N\elt\{1,2,3\}\) and \(M_1,M_2\elt\{0,1,2,3,4\}\),
with the symbolic CPTs as described in \exref{telescope-exercise}.
Using the enumeration algorithm (\figref{enumeration-algorithm} on \pgref{enumeration-algorithm}), calculate the probability
distribution \(\pv(N\given M_1\eq 2,M_2\eq 2)\).
\end{uexercise} 
% id=14.6 section=14.4

\begin{figure}[tbp]
\figboxspacenew{3pt}{figures/telescope-nets.eps}
\caption{Three possible networks for the telescope problem.}
\label{telescope-nets-figure}
\end{figure} 
% id=14.7 section=14.4

\begin{figure}[h]
\figboxnew{figures/politics.eps}%%<<<upgrade figure]]
\caption{A simple Bayes net with
Boolean variables \(B\eq \J{BrokeElectionLaw}\), \(I\eq \J{Indicted}\), 
\(M\eq \J{PoliticallyMotivatedProsecutor}\), \(G\eq \J{FoundGuilty}\), 
\(J\eq \J{Jailed}\).}
\label{politics-figure}
\end{figure} 
% id=14.11 section=14.4

\begin{uexercise}
Consider the Bayes net shown in \figref{politics-figure}.
\begin{enumerate}
\item Which of the following are asserted by the network {\em structure}?
\begin{enumerate}
\item \(\pv(B,I,M) = \pv(B)\pv(I)\pv(M)\).
\item \(\pv(J\given G) = \pv(J\given G,I)\).
\item \(\pv(M\given G,B,I) = \pv(M\given G,B,I,J)\).
\end{enumerate}
\item Calculate the value of \(P(b,i,\lnot m,g,j)\).
\item Calculate the probability that someone goes to jail given that they
broke the law, have been indicted, and face a politically motivated prosecutor.
\item A \term{context-specific independence}\tindex{independence!context-specific} (see \pgref{CSI-page})
allows a variable to be independent of some of its parents given certain values of others. In addition to the usual
conditional independences given by the graph structure, what context-specific independences
exist in the Bayes net in \figref{politics-figure}?
\item Suppose we want to add the variable \(P\eq \J{PresidentialPardon}\)
to the network; draw the new network and 
briefly explain any links you add.
\end{enumerate}
\end{uexercise} 
% id=14.12 section=14.4

\begin{iexercise}
Consider the Bayes net shown in \figref{politics-figure}.
\begin{enumerate}
\item Which, if any, of the following are asserted by the 
network {\em structure} (ignoring the CPTs for now)?
\begin{enumerate}
\item \(\pv(B,I,M) = \pv(B)\pv(I)\pv(M)\).
\item \(\pv(J\given G) = \pv(J\given G,I)\).
\item \(\pv(M\given G,B,I) = \pv(M\given G,B,I,J)\).
\end{enumerate}
\item Calculate the value of \(P(b,i,m,\lnot g,j)\).
\item Calculate the probability that someone goes to jail given that they
broke the law, have been indicted, and face a politically motivated prosecutor.
\item A \term{context-specific independence}\tindex{independence!context-specific} (see \pgref{CSI-page})
allows a variable to be independent of some of its parents given certain values of others. In addition to the usual
conditional independences given by the graph structure, what context-specific independences
exist in the Bayes net in \figref{politics-figure}?
\item Suppose we want to add the variable \(P\eq \J{PresidentialPardon}\)
to the network; draw the new network and 
briefly explain any links you add.
\end{enumerate}
\end{iexercise} 
% id=14.12 section=14.4



\begin{exercise}[VE-exercise]
Consider the variable elimination algorithm  in
\figref{elimination-ask-algorithm} (\pgref{elimination-ask-algorithm}). 
\begin{enumerate}
\item \secref{exact-inference-section} applies
variable elimination to the query 
\[
  \pv(\J{Burglary}\given \J{JohnCalls}\eq \J{true},\J{MaryCalls}\eq \J{true})\ .
\]
Perform the calculations indicated and check that the answer is correct.
\item Count the number of arithmetic operations performed,
and compare it with the number performed by the enumeration algorithm.
\item Suppose a network has the form of a {\em chain}: a sequence of
Boolean variables \(X_1,\ldots, X_n\) where \(\Parents(X_i)\eq \{X_{i-1}\}\) for
\(i\eq 2,\ldots,n\). What is the complexity of computing \(\pv(X_1\given X_n\eq
\J{true})\) using enumeration? Using variable elimination?
\item Prove that the complexity of running variable elimination on a
polytree network is linear in the size of the tree for any variable ordering
consistent with the network structure.
\end{enumerate}
\end{exercise} 
% id=14.18 section=14.4

\begin{exercise}[bn-complexity-exercise]
Investigate the complexity of exact inference in general Bayesian networks:
\begin{enumerate}
\item Prove that any 3-SAT problem can be reduced to exact inference
in a Bayesian network constructed to represent the particular problem
and hence that exact inference is NP-hard. ({\em Hint}: Consider a network
with one variable for each proposition symbol, one for each clause,
and one for the conjunction of clauses.)
\item The problem of counting the number of satisfying assignments for
a 3-SAT problem is \#P-complete. Show that exact
inference is at least as hard as this.
\end{enumerate}
\end{exercise} 
% id=14.19 section=14.4


%%%% 14.5: Approximate Inference in Bayesian Networks (4 exercises, 3 labelled)
%%%% ==========================================================================

\begin{exercise}[primitive-sampling-exercise]
Consider the problem of generating a random sample from a specified
distribution on a single variable. Assume you have a random number
generator that returns a random number uniformly
distributed between 0 and 1.
\begin{enumerate}
\item Let \(X\) be a discrete variable with \(P(X\eq x_i)\eq p_i\)
for \(i\elt\{1,\ldots,k\}\). The \newterm{cumulative distribution}\ntindex{distribution!cumulative} of
\(X\) gives the probability that \(X\elt\{x_1,\ldots,x_j\}\) for each
possible \(j\). (See also \appref{math-appendix}.) Explain how to calculate the cumulative distribution in
\(O(k)\) time and how to generate a single sample of \(X\)
from it. Can the latter be done in less than
\(O(k)\) time?

\item Now suppose we want to generate \(N\) samples of \(X\), where \(N\gg k\).
Explain how to do this with an expected run time per sample that is
{\em constant} (i.e., independent of \(k\)).
\item Now consider a continuous-valued variable with a parameterized
distribution (e.g., Gaussian). How can samples be generated from such
a distribution? 
\item Suppose you want to query a continuous-valued variable and you
are using a sampling algorithm such as \noprog{LikelihoodWeighting} to
do the inference. How would you have to modify the query-answering
process?
\end{enumerate}
\end{exercise} 
% id=14.20 section=14.5

\begin{exercise}
Consider the query \(\pv(\J{Rain}\given \J{Sprinkler}\eq
\J{true},\J{WetGrass}\eq \J{true})\) in \figref{rain-clustering-figure}(a) (\pgref{rain-clustering-figure})
and how Gibbs sampling can answer it.
\begin{enumerate}
\item How many states does the Markov chain have?
\item Calculate the \term{transition matrix}\tindex{transition matrix}
\(\mbf{Q}\) containing \(\transition{\mbf{y}}{\mbf{y}'}\) for all \(\mbf{y}\),
\(\mbf{y}'\).
\item What does \(\mbf{Q}^2\), the square of the transition matrix, represent?
\item What about \(\mbf{Q}^n\) as \(n\to \infty\)?
\item Explain how to do probabilistic inference in Bayesian networks,
assuming that \(\mbf{Q}^n\) is available. Is this a practical way to do inference?
\end{enumerate}
\end{exercise} 
% id=14.22 section=14.5

\begin{exercise}[gibbs-proof-exercise]
This exercise explores the stationary distribution for Gibbs sampling methods.
\begin{enumerate}
\item The convex composition \([\alpha, q_1; 1-\alpha, q_2]\) of \(q_1\) and \(q_2\) is a transition probability distribution that first chooses one of \(q_1\) and \(q_2\) with probabilities \(\alpha\) and \(1-\alpha\), respectively, and then applies whichever is chosen.
Prove that if \(q_1\) and \(q_2\) are in detailed balance with \(\pi\),
then their convex composition is also in detailed balance with \(\pi\). ({\em Note}: this result justifies a variant of \noprog{Gibbs-Ask} in which variables are chosen at random rather than sampled in a fixed sequence.)
\item Prove that if each of \(q_1\) and \(q_2\) has \(\pi\) as its stationary distribution,
then the sequential composition \(q \eq q_1 \circ q_2\) also has \(\pi\) as its stationary distribution.
\end{enumerate}
\end{exercise} 
% id=14.23 section=14.5

\begin{exercise}[MH-exercise]
The \newtermi{Metropolis--Hastings} algorithm is a member of the MCMC family; as such, it is designed to generate samples \(\x\) (eventually) according to target probabilities \(\pi(\x)\). (Typically we are interested in sampling from \(\pi(\x)\eq P(\x\given \e)\).)
Like simulated annealing, Metropolis--Hastings operates in two stages. First,
it samples a new state \(\x'\) from a \newtermi{proposal distribution} \(q(\x'\given \x)\), given the current state \(\x\). Then, it probabilistically accepts or rejects \(\x'\) according to
the \newterm{acceptance probability}
\[
   \alpha(\x'\given \x) = \min\ \left(1,\frac{\pi(\x')q(\x\given \x')}{\pi(\x)q(\x'\given \x)}  \right)\ .
\]
If the proposal is rejected, the state remains at \(\x\).
\begin{enumerate}
\item Consider an ordinary Gibbs sampling step for a specific variable \(X_i\). Show that this step, considered as a proposal, is guaranteed to be accepted by Metropolis--Hastings.
(Hence, Gibbs sampling is a special case of Metropolis--Hastings.)
\item Show that the two-step process above, viewed as a transition probability distribution, is in detailed balance with \(\pi\).
\end{enumerate}
\end{exercise} 
% id=14.24 section=14.5


%%%% 14.6: Relational and First-Order Probability Models (1 exercises, 1 labelled)
%%%% =============================================================================

\begin{exercise}[soccer-rpm-exercise]\prgex%
Three soccer teams \(A\), \(B\), and \(C\), play each other once. Each match
is between two teams, and can be won, drawn, or lost. Each team has a
fixed, unknown degree of quality---an integer ranging from 0 to
3---and the outcome of a match depends probabilistically on the
difference in quality between the two teams.
\begin{enumerate}
\item Construct a relational probability model to describe this
domain, and suggest numerical values for
all the necessary probability distributions.
\item Construct the equivalent Bayesian network for the three matches.
\item Suppose that in the first two matches \(A\) beats \(B\) and draws with \(C\).
Using an exact inference algorithm of your choice, compute the posterior
distribution for the outcome of the third match.
\item Suppose there are \(n\) teams in the league and we have the results
for all but the last match. How does the complexity of predicting
the last game vary with \(n\)?
\item Investigate the application of MCMC to this problem. How quickly
does it converge in practice and how well does it scale?
\end{enumerate}
\end{exercise} 
% id=14.25 section=14.6




