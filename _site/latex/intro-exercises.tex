\noindent These exercises are intended to stimulate discussion, and 
some might be set as term projects.  Alternatively, preliminary 
attempts can be made now, and these attempts can be reviewed after 
the completion of the book.

%%%% 1.1: What Is AI? (4 exercises, 0 labelled)
%%%% ==========================================

\begin{exercise}
    Define in your own words: (a) intelligence, (b) artificial 
    intelligence, (c) agent, (d) rationality, (e) logical reasoning.
\end{exercise} 
% id=1.0 section=1.1

\begin{uexercise}\libex
Read Turing's original paper on AI \cite{Turing:1950}. In the paper,
he discusses several objections to his proposed enterprise
and his test\index{Turing Test} for intelligence. Which objections
still carry weight? Are his refutations valid? Can you think of
new objections arising from developments since he wrote the paper? In
the paper, he predicts that, by the year 2000, a computer will have a
30\% chance of passing a five-minute Turing Test with an unskilled
interrogator. What chance do you think a computer would have today?
In another 50 years?
\end{uexercise} 
% id=1.1 section=1.1.1

\begin{iexercise}\libex
    Every year the Loebner Prize\index{Loebner Prize} is awarded to the 
    program that comes 
    closest to passing a version of the Turing Test. Research and 
    report on the latest winner of the Loebner prize.  What techniques 
    does it use? How does it advance the state of the art in AI?
\end{iexercise} 
% id=1.2 section=1.1.1

\begin{exercise}
    Are reflex actions (such as flinching from a hot stove) rational? Are they intelligent?%
\end{exercise}%
% id=1.12 section=1.1.4


%%%% 1.2: The Foundations of Artificial Intelligence (12 exercises, 0 labelled)
%%%% ==========================================================================

\begin{iexercise}
There are well-known classes of problems that are
intractably\index{intractability} difficult for computers, and other
classes that are provably undecidable\index{undecidability}.
Does this mean that AI is impossible?
\end{iexercise} 
% id=1.3 section=1.2.2

\begin{uexercise}
Suppose we extend Evans's\nindex{Evans, T.~G.} \system{Analogy}
program so that it can score 200 on a standard 
IQ test\index{IQ test}. Would we then have a program more
intelligent than a human?
Explain.
\end{uexercise} 
% id=1.4 section=1.2.5

\begin{exercise}
The neural structure of the sea slug {\em Aplysia\/} has been widely studied 
(first by Nobel Laureate Eric Kandel) because it has only about 20,000
neurons, most of them large and easily manipulated.  
Assuming that the cycle
time for an {\em Aplysia\/} neuron is roughly the same as for a human neuron, how
does the computational power, in terms of memory updates per second, 
compare with the high-end computer described in \figref{computer-brain-table}?
\end{exercise} 
% id=1.5 section=1.2.4

\begin{exercise}
    How could introspection---reporting on one's inner thoughts---be 
    inaccurate?   Could I be wrong about what I'm thinking?  Discuss.
\end{exercise} 
% id=1.6 section=1.2.5

\begin{uexercise}
To what extent are the following computer systems instances of artificial
intelligence:
\begin{itemize}
\item Supermarket bar code scanners.
\item Web search engines.
\item Voice-activated telephone menus.
\item Internet routing algorithms that respond dynamically to the state
of the network.
\end{itemize}
\end{uexercise} 
% id=1.7 section=1.2.6

\begin{iexercise}
To what extent are the following computer systems instances of artificial
intelligence:
\begin{itemize}
\item Supermarket bar code scanners.
\item Voice-activated telephone menus.
\item Spelling and grammar correction features in Microsoft Word.
\item Internet routing algorithms that respond dynamically to the state
of the network.
\end{itemize}
\end{iexercise} 
% id=1.7 section=1.2.6

\begin{exercise}
Many of the computational models of cognitive activities that have been 
proposed involve quite complex mathematical operations, such as convolving
an image with a Gaussian or finding a minimum of the entropy function.  
Most humans (and certainly all animals) never learn this kind of mathematics
at all, almost no one learns it before college, and almost no one can compute
the convolution of a function with a Gaussian in their head.  What sense does 
it make to say that the ``vision system'' is doing this kind of mathematics,
whereas the actual person has no idea how to do it?
\end{exercise} 
% id=1.9 section=1.2.2

\begin{iexercise}
Some authors have claimed that perception\index{perception} and motor
skills are the most important part of intelligence, and that
``higher level'' capacities are necessarily parasitic---simple add-ons
to these underlying facilities.  Certainly, most of
evolution\index{evolution} and a large part of the brain have been
devoted to perception and motor skills, whereas AI has found tasks
such as game playing and logical inference to be easier, in many ways,
than perceiving and acting in the real world. Do you think that AI's
traditional focus on higher-level cognitive abilities is misplaced?
\end{iexercise} 
% id=1.10 section=1.2.1

\begin{uexercise}
    Why would evolution tend to result in systems that act rationally?
    What goals are such systems designed to achieve?
\end{uexercise} 
% id=1.11 section=1.2

\begin{exercise}
Is AI a science, or is it engineering? Or neither or both? Explain.
\end{exercise} 
% id=1.13 section=1.2

\begin{exercise}
``Surely computers cannot be intelligent---they can do only what their
programmers tell them.'' Is the latter statement true, and does it imply the
former?
\end{exercise} 
% id=1.14 section=1.2.1

\begin{exercise}
``Surely animals cannot be intelligent---they can do only what their
genes tell them.''  Is the latter statement true, and does it imply the
former?
\end{exercise} 
% id=1.15 section=1.2.1

\begin{exercise}
``Surely animals, humans, and computers cannot be intelligent---they 
can do only what their constituent atoms are told to do by the laws 
of physics.''  Is the latter statement 
true, and does it imply the former?
\end{exercise} 
% id=1.16 section=1.2.1


%%%% 1.4: The State of the Art (2 exercises, 0 labelled)
%%%% ===================================================

\begin{exercise}\libex
Examine the AI literature to discover whether  the following
tasks can currently be solved by computers:
\begin{enumerate}
\item Playing a decent game of table tennis (Ping-Pong)\index{table tennis}\index{ping-pong}.
\item Driving in the center of Cairo, Egypt.
\item Driving in Victorville, California.
\item Buying a week's worth of groceries at the market.
\item Buying a week's worth of groceries on the Web.
\item Playing a decent game of bridge\index{bridge (card game)} at a competitive level.
\item Discovering and proving new mathematical theorems\index{theorem proving!mathematical}.
\item Writing an intentionally funny story.
\item Giving competent legal\index{legal reasoning} advice in a specialized area of law.
\item Translating\index{machine translation} spoken English\index{English} into spoken Swedish\index{Swedish} in real time.
\item Performing a complex surgical operation.
\end{enumerate}
For the currently infeasible tasks, try to find out what the
difficulties are and predict  when, if ever, they will be overcome.
\end{exercise} 
% id=1.8 section=1.4

\begin{exercise}
Various subfields of AI have held contests by defining a standard task
and inviting researchers to do their best.  Examples include the DARPA
Grand Challenge for robotic cars, the International Planning
Competition, the Robocup robotic soccer league, the TREC information
retrieval event, and contests in machine translation and speech
recognition. Investigate five of these contests and describe the
progress made over the years. To what degree have the contests
advanced the state of the art in AI?  To what degree do they hurt the field
by drawing energy away from new ideas?
\end{exercise} 
% id=1.17 section=1.4




