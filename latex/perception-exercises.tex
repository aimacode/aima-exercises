%%%% 24.1: Image Formation (3 exercises, 0 labelled)
%%%% ===============================================

\begin{exercise}
In the shadow of a tree with a dense, leafy canopy, one sees a number of 
light spots. Surprisingly, they all appear to be circular. Why? After all,
the gaps between the leaves through which the sun shines are not
likely to be circular.
\end{exercise} 
% id=24.0 section=24.1

\begin{uexercise}
Consider a picture of a white sphere floating in front of a black backdrop.  The image curve separating white pixels from
black pixels is sometimes called the ``outline'' of the sphere.  Show that the outline of a sphere, viewed in
a perspective camera, can be an ellipse.  Why do spheres not look like ellipses to you?
\end{uexercise} 
% id=24.1 section=24.1

\begin{exercise}
Consider an infinitely long cylinder of radius \(r\) oriented with its
axis along the \(y\)-axis. The cylinder has a Lambertian\index{Lambertian surface} surface
and is viewed by a camera along the positive \(z\)-axis.  What will you
expect to see in the image  if the cylinder is
illuminated by a point source at infinity located on the positive
\(x\)-axis?  Draw the contours of constant brightness in
the projected image. Are the contours of equal brightness uniformly spaced?
\end{exercise} 
% id=24.2 section=24.1


%%%% 24.2: Early Image-Processing Operations (1 exercises, 0 labelled)
%%%% =================================================================

\begin{exercise}
Edges in an image can correspond to a variety of events in a scene.
Consider \figref{illuminationfigure} (\pgref{illuminationfigure}), and
assume that it is a picture of a real three-dimensional
scene. Identify ten different brightness edges in the image, and for
each, state whether it corresponds to a discontinuity in (a) depth,
(b) surface orientation, (c) reflectance, or (d) illumination.
\end{exercise} 
% id=24.3 section=24.2.1


%%%% 24.4: Reconstructing the 3D World (3 exercises, 0 labelled)
%%%% ===========================================================

\begin{exercise}
A stereoscopic system is being contemplated for terrain mapping. It
will consist of two CCD\index{charge-coupled device}\index{CCD
(charge-coupled device)} cameras, each having \({512}\times {512}\)
pixels on a 10 cm \(\times\) 10 cm square sensor.  The lenses to be
used have a focal length of 16 cm, with the focus fixed at
infinity. For corresponding points (\(u_1,v_1\)) in the left image and
(\(u_2,v_2\)) in the right image, \(v_1=v_2\) because the \(x\)-axes
in the two image planes are parallel to the epipolar lines---the lines
from the object to the camera.  The optical axes of the two cameras
are parallel. The baseline between the cameras is 1 meter.
\begin{enumerate}
\item If the nearest distance to be measured is 16 meters, what is the largest
disparity that will occur (in pixels)?

\item  What is the distance resolution at 16 meters, due to the pixel spacing?

\item  What distance corresponds to a disparity of one pixel?
\end{enumerate}
\end{exercise} 
% id=24.4 section=24.4.2


\begin{uexercise}
Which of the following are true, and which are false?
\begin{enumerate}
\item Finding corresponding points in stereo images is the easiest phase of
the stereo depth-finding process.
\item Shape-from-texture can be done by projecting a grid of light-stripes
onto the scene.
%\item The Huffman--Clowes labelling scheme can deal with all polyhedral
%objects.
%\item Generalized cylinders provide a viewpoint--independent
%representation of volume elements.
%\item In line drawings of curved objects, the line label can change from one 
%end of the line to the other. 
\item  Lines with equal lengths in the scene  always project to equal
lengths in the image.
\item  Straight lines in the image necessarily correspond to straight lines 
in the scene.
\end{enumerate}
\end{uexercise} 
% id=24.7 section=24.4

\begin{iexercise}
Which of the following are true, and which are false?
\begin{enumerate}
\item Finding corresponding points in stereo images is the easiest phase of
the stereo depth-finding process.
%\item The Huffman--Clowes labelling scheme can deal with all polyhedral
%objects.
%\item Generalized cylinders provide a viewpoint--independent
%representation of volume elements.
%\item In line drawings of curved objects, the line label can change from one 
%end of the line to the other. 
\item In stereo views of the same scene, greater accuracy is obtained in the
depth calculations if the two camera positions are farther apart.
\item  Lines with equal lengths in the scene  always project to equal
lengths in the image.
\item  Straight lines in the image necessarily correspond to straight lines 
in the scene.
\end{enumerate}
\end{iexercise} 
% id=24.7 section=24.4


%% \ignore{\begin{exercise}
%% Label the line drawing in \figref{l-shape-figure}, assuming that the
%% outside edges have been labeled as occluding and that all vertices
%% are trihedral. Do this by a backtracking\index{search!backtracking}\index{backtracking} algorithm that examines the
%% vertices in the order \(A\), \(B\), \(C\), and \(D\), picking at each stage a choice
%% consistent with previously labeled junctions and edges. Now try 
%% the order \(B\), \(D\), \(A\), and \(C\).
%% \end{exercise}



%% \begin{figure}[ht]
%% %%%%\epsfxsize=2.6in
%% \figboxnew{figures/l-shape.eps}
%% \caption{A drawing to be labeled, in which all vertices are trihedral.}
%% \label{l-shape-figure}
%% \end{figure}
%% }



\ignore{
\begin{exercise}
Show that convolution\index{convolution} with a given function \(f\) commutes with
differentiation; that is, show that
\((f \ast g)' = f \ast g'\ \).
\end{exercise}
}

%% \begin{exercise}\label{cross-ratio-exercise}%
%% In \figref{cross-ratio-figure}, physically measure the cross ratio of
%% the points {\mathdelim}ABCD{\mathdelim} as well as of the points {\mathdelim}A'B'C'D'{\mathdelim}. Are they equal?
%% \end{exercise}

\ignore{
\begin{exercise}
Suppose we wish to use the alignment algorithm in an industrial
situation in which flat parts move along a conveyor belt and are
photographed by a camera vertically above the conveyor belt.  The pose
of the part is specified by three variables---one for the rotation and
two for the two-dimensional position. This simplifies the problem and
the function \prog{Find-Transform} needs only two pairs of
corresponding image and model features to determine the pose.
Determine the worst-case complexity of the alignment procedure.
\end{exercise}
}



\ignore{
\begin{exercise}
\figref{lane-marker-figure} is taken from the point of
view of a car in the exit lane of a freeway. Two cars
are visible in the lane immediately to the left.
What reasons does the viewer have to conclude that one is
closer than the other?
\end{exercise}
}

\pagebreak

\begin{figure}[ht]
%%%%\epsfxsize=0.48\maxfigwidth
\figboxnew{figures/bottle-stereo.eps}
\caption{Top view of
a two-camera vision system observing a bottle with a wall behind it.}
\label{bottle-figure}
\end{figure} 
% id=24.6 section=24.4.3

\begin{exercise} (Courtesy of Pietro Perona.)
\figref{bottle-figure} shows two cameras at X
and Y
observing a scene. Draw the image seen at each camera,
assuming that all named points are in the same horizontal plane.
What can be concluded from these two images about the relative
distances of points A, B, C, D, and E from the camera baseline, and
on what basis?
\end{exercise} 
% id=24.5 section=24.4.3

